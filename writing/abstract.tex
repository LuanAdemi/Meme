\abstract{
Network devices such as routers, switches, and firewalls forward traffic based on entries in their local forwarding tables. Although these forwarding tables conventionally make decisions based on a packet header field such as a destination address, rules that match on a sequence or a set and make forwarding decisions based on elements in that set can enable richer network policies. For example, devices at the edge of a network could add a tag to each packet that encodes a set of middleboxes to traverse, a set of egress locations, or a set of anycast destinations; simpler devices in the core of the network could then forward packets based on this tag. 

Unfortunately, naive construction of these tags can create forwarding tables that grow quadratically with the number of elements in the set or sequence--prohibitive for commodity network devices. In this paper, we present a compression algorithm that makes such encodings practical. The compression algorithm encodes sequences or sets (e.g., middlebox service chains, lists of next-hop network devices) in a compact tag that fits in a small packet header field. Our evaluation shows that the compression technique can encode common forwarding sequences for large networks using only a few bits and that the number of forwarding rules grows linearly with the number of elements in the set or sequence.
}

