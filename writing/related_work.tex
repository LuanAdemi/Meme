\section{Related Work} 
\label{sec:related}

\paragraph{Software-defined Internet exchange points.} The first
SDX~\cite{sdx} prototype in 2014 used a flat tag to identify the set of
valid next hops (\ie, BGP neighbors) that can direct traffic to a
destination prefix to avoid installing forwarding table entries for each
destination IP prefix in the IXP switch; this prototype SDX controller
proactively installed rules to decode each tag.  The follow-up
industrial scale iSDX~\cite{isdx} project reduced forwarding-table size
and rule churn to make operation at large IXPs practical.  The iSDX uses
the technique for encoding sets of attributes described in
Section~\ref{sec:flextag_encoding}, although our iSDX paper did not
describe the encoding scheme in much detail.  In addition, the iSDX
design supports neither the encoding of {sequences} of attributes
(Section~\ref{sec:ordering}) nor the optimizations possible with
variable-length group identifiers (Section~\ref{sec:identifiers}).

\paragraph{Service chaining with flat tags.} Recent works on traffic
steering through middleboxes uses a flat tag to identify the service
chain each packet should traverse.  FlowTags~\cite{flowtags} is a
prominent example, where the first middlebox in a path tags the packet
with the rest of the service chain. Rules to decode each tag are
installed reactively when switches see a new tag for the first time.  In
contrast, the encoding scheme we present in this work results in fewer
rules, allowing for proactive rule installation.
Incorporating the encoding mechanism that we present in this paper would
make FlowTags scale to larger sets and sequences of middleboxes.

\paragraph{Encoding end-host attributes in IP addresses.}
Alpaca~\cite{alpaca} embeds the attributes of end hosts into the packet
header with the goal of easing network policy enforcement.  Alpaca
focuses on a special case where attributes are embedded in the low-order
bits of the IP address. This mechanism requires a unique attribute
encoding for each host.  In addition, Alpaca does not support the
encoding of sequences of attributes.  As such, the encoding algorithm we
present can solve the problem introduced in Alpaca, but the reverse is
not true.

\paragraph{Forwarding with flat tags.}
MPLS~\cite{mpls} is a well-established source-routing protocol for forwarding 
packets by writing and matching on flat labels in packet headers. MPLS can
 be used for tunneling unsupported protocols through relatively simple 
 match-action switches. However, MPLS tags only instruct the packet 
 how to reach the next node in a path by default. To use MPLS across 
 multiple hops in succession, MPLS performs label swaps, where the tag is
  swapped out for a new tag. MPLS can extended to support segment routing~\cite{segmentrouting}, 
  by having each label represent a path segment, rather than a single hop. 
While powerful, an issue with MPLS is that labels are often stacked, causing 
packets to have variable length as they are forwarded. The Path Switching work~\cite{pathswitching} presents an alternate to MPLS for source routing which
has the advantage of encoding forwarding information in a fixed amount 
of existing space in the packet headers. 

\paragraph{Encoding sets concisely in Bloom filters.} A Bloom
filter~\cite{Bloom} is a common data structure that represents a set of
items with a fixed amount of memory. A Bloom filter supports membership
queries but suffers from false positives, where some elements can be
wrongly reported as members of the set. In contrast, the encoding
mechanism we develop has no false positives.  Although minimizing the
tag width is an key property of our scheme, the memory for a Bloom
filter is several times larger than the number of elements---10-20 times
for a false-positive probability of 0.01--1\%. Bloom filters also
require each value to be hashed into a large bit vector, which might not
be possible in some switch architectures.

\paragraph{Efficient coding using prefix codes.} Variable-length prefix
codes have been used for various applications. The seminal work of
Huffman~\cite{Huffman} describes an algorithm for an optimal selection
of prefix codes for lossless compression of source symbols. The
selection minimizes the encoding length by using fewer bits for common
symbols, achieving results close to lower bounds from information
theory. More recently, prefix codes were suggested as a way to encode
paths, while reducing the maximum length of any encoded
path~\cite{PathEncoding}. A similar approach was suggested for the
encoding for fixed-width memories~\cite{FixedMemories}. In all of these
schemes, the encoding concatenates the codes of the attributes and thus
is often long when the number of attributes is large. In contrast, the
encoding we develop
combines input sets or sequences with common attributes (or
common orderings of attributes) and uses prefix codes for the
group identifiers to further reduce tag size.
