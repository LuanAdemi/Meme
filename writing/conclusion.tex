\section{Conclusion} \label{sec:conclusion}

Many network architectures and mechanisms, from SDN-based exchange
points to middlebox service chains, rely on mechanisms to encode sets of
sequences in forwarding equivalence classes. Previous work has generally
encoded each FEC with a flat tag, which is amenable to exact matching
but scales poorly as the size of a set or the number of unique orderings
increases. In this paper, we propose a mechanism that takes advantage of
the ability of commodity switches to perform wildcard matching
on arbitrary packet header fields; this capability, which has been
enabled by protocols such as OpenFlow~1.3, facilitates more efficient
encodings that allow aggregation of FECs that share commonality, through
prefix-based encoding. 

Our evaluation for two deployment scenarios---for service chaining and
an SDN-based IXP---demonstrates that a wildcard-based encoding can
reduce the number of forwarding table entries in each switch by several
orders of magnitude, given typical levels of redundancy in the sets or
sequences that are typical in forwarding policies for these
scenarios. The encoding can capture both unordered sets and ordered
sequences; it can also efficiently capture rare perturbations to
sequences that appear more frequently.


The encoding algorithm we present reduces forwarding table size, at the
cost of larger tags. A possible avenue for future work involves
exploring this tradeoff more thoroughly, as well as exploring
alternatives ways to handle attribute sequences that have conflicting
attribute orderings.  Every application that we presented relies on a
different method to attach tags to packets. To make it easier to tag
packets for arbitrary network deployment scenarios, we plan to
investigate more general techniques for tagging packets that could apply
to a wider range of use cases and possibly express even more expressive
forwarding semantics. 