\section{Conclusion} \label{sec:conclusion}

Although our scheme improves upon the switch memory usage of existing tagging approaches, it uses larger tags to do so. As a future work, it would be interesting to explore better memory-width tradeoffs, as well as better ways to handle attribute sequences which have conflicting attribute orderings.

Each technique presented here has a customized method for attaching tags to packets. To make it easier to deploy tagging schemes to new applications, investigating application-agnostic techniques for tagging packets could open up a whole new range of use cases. 


We have shown that our tagging scheme can improve upon flat tagging by orders of magnitude if there is enough redundancy in the data that tags convey. The scheme is versatile, able to encode both unordered sets and ordered sequences, as well as able to handle scenarios where the ordering is not perfect. 

Our encoding scheme is possible because of the growing popularity of flexible switching hardware, such as switches that support OpenFlow and the newly emerging standard P4 for programmable switches. The continuing expansion of switch capabilities allows exploration of new techniques and applications for attaching increasingly complex information directly to packet flows, and we hope our work is not the last to explore this. 