\section{Introduction}


Routing flows through a network is super fast and easy. But as networks evolve, routing is becoming less simple than just sending packets along a shortest path to their destination. Depending upon their header fields, they may be redirected through middleboxes, subject to security policies, or have multiple destinations. However, not every combination of header fields is treated differently. Two flows that are treated identically by the network can be grouped into a single \textit{Forwarding Equivalence Class}, or FEC for short. If the number of FECs is small, the network can behave correctly using only a small amount of information about how to handle each equivalence class. However, classifying packets to FECs can be expensive, using a large amount of pattern matches on the headers. To avoid repeated classifications, the result could be attached to the packet in a way that network nodes understand. It is a theme amongst solutions[ref:flowtags][ref:sdx][ref:mpls] to implicitly build a table of all seen FECs and then tag each packet with an index into that table. If network nodes have knowledge of the table, they need only read the index to decide their behavior. An example application is virtual circuit switching, where each packet receives a tag mapping it to a path through the network. 

This index tagging technique was shaped by the capabilities of commodity switches, which were only able to perform exact matching on fields of supported header formats. Recently, commodity switches have emerged which have extended capabilities. Openflow 1.3 [cite:openflow13] introduced the ability to perform a limited amount of wildcard matching on fields, and the rise of protocol-independent programmable switches like those defined by the P4 specification [cite:p4] allow arbitrary header formats to be created which can be read and written in arbitrary ways. With these advances comes the opportunity to rethink the way FEC information is attached to packets beyond simple index tagging. 

Indeed, FEC index tagging solutions can run into memory scaling issues. 
Existing solutions make no attempt to assign tags in a way that allows network nodes to aggregate classes. In the case of virtual circuit switching with $N$ distinct circuits, it can be the case that $\frac{N}{2}$ circuits begin with nodes ABC, and $\frac{N}{2}$ begin with ABD. In the worst case, Node B will have to compare every packet's tag to at least $\frac{N}{2}$ tags to decide whether to forward to C or D, because it must exhaustively rule out one of the two classes. Fundamentally, the issue is that each equivalence class is characterized by a set of attributes, yet two equivalence classes which differ by only one attribute receive arbitrary indices. For virtual circuits, the attributes are a list of hops, and the difference between two circuits can be a divergence at the end of the circuit.



%In the case of virtual circuit switching, the number of possible circuits can be exponential in the size of the network. Switches need to be programmed to react to every tag they may see, resulting in exponential memory usage. It may be the case that the majority of equivalence classes that a switch sees take identical egress ports, yet traditional tagging is unable to take advantage of this redundancy. Fundamentally, an equivalence class can be characterized by a set of attributes where that set is unique to that class. If two circuits differ by only a single hop, their attribute sets are unique and they are assigned different tags. Traditional solutions make no attempt to convey the similarity of the two sets in the tags, which could allow switches to be programmed with a single rule that reacts to both tags. We refer to these solutions as \textit{flat tagging} solutions. 




%Paragraph 3: "In this paper, we show that ...". This is the key paragraph in the intro - you summarize, in one paragraph, what are the main contributions of your paper given the context you have established in paragraphs 1 and 2. What is the general approach taken? Why are the specific results significant? This paragraph must be really really good. If you can't "sell" your work at a high level in a paragraph in the intro, then you are in trouble. As a reader or reviewer, this is the paragraph that I always look for, and read very carefully.

In this paper, we show how the set of attributes that define an equivalence class can be embedded in the assigned tag. The attributes of any tag can then be individually read using a small set of wildcard rules. These attribute-carrying tags, or a-tags, can decrease the amount of memory used by core switches by up to two orders of magnitude compared to index tagging solutions. We show how these tags can be used in several real applications, including one which has seen real-world deployment. Additionally, we show how these tags not only improve existing solutions, but also open new applications. We perform evaluations on both real and synthetic datasets for our proposed applications.



%Paragraph 4: At a high level what are the differences in what you are doing, and what others have done? Keep this at a high level, you can refer to a future section where specific details and differences will be given. But it is important for the reader to know at a high level, what is new about this work compared to other work in the area.



%Paragraph 5: "The remainder of this paper is structured as follows..." Give the reader a roadmap for the rest of the paper. Avoid redundant phrasing, "In Section 2, In section 3, ... In Section 4, ... " etc.

The remainder of this paper is structured as follows. In Section 2, we give some area background and a few motivating applications. In Sections 3 and 5, we build our solution in two stages. In sections 4 and 6, we discuss how to optimize the two stages of our solution. In section 7, we evaluate the solution over both real and synthetic data sets. The paper concludes with whatever filler we have space for.






%\subsection{Common Ground}
%
%Although a diverse set of applications, each of these problems fits a common framework. As a packet enters a local area network, it is classified as belonging to some category of traffic. Associated with this category of traffic is a sequence or set. During classification, this sequence is somehow attached to the packet header.
%All three of these examples have a common framework: Packets are fit into categories as they enter a local area network, and associated with each category is some sequence of information. These sequences can be middleboxes, hosts, next-hop switches, or something else entirely. The sequence could be ordered, as in the case of middlebox paths, or unordered as in the case of feasible next-hops. In each case, the sequence is read by the local network switches to determine which direction to route the traffic. We refer to this reading of information as \textit{membership testing}, because routing choices are decided based upon which hosts or middleboxes are members of the sequence. 

%\subsection{Forwarding Table Matching On Tags}
%
%
%For this scheme of attaching information sets to packets to be feasible with commodity switches, it must be possible for membership testing to be implemented in the forwarding tables of switches. In switch TCAM tables, rules are are comparisons between a fixed string and the packet header, where the fixed strings are over the alphabet $\{0,1,*\}$. $0$ and $1$ are specific bit values, and $*$ denotes "don't care". We say that a packet header matches a string if for every bit in the header, either the bits are equal or one is a wildcard. Example usages include exact matches (strings with no wildcards), prefix matches (strings that end in wildcards), and reading of individual bits (strings with only one non-wildcard). 
%
%Various applications benefit from the use of TCAM 
%
% (TODO: cite some tagging works like flowtags and the original SDX?) have solved the problem of associating packets with information sets by generating a tag for each unique information set and repurposing one of the fields in the header for the tag. In the FlowTags work, each middlebox path had its own set of tags, which could fit into the IP Fragment Identification field. To determine if middlebox $X$ is the next-hop, switches must compare the tag to every tag which has $X$ as a next-hop using TCAM exact matches. This can result in a TCAM entry count exponential in the number of bits in a tag. In the SDX work, each set of next-hops had a unique tag which was placed in the destination mac field. Again, to determine if next-hop $X$ is correct, the tag must be compared against every tag which contains $X$ using exact TCAM matches. 
%
%In both works, the number of bits required can be quite small, but membership tests are expensive, requiring rules exponential in the tag size. TCAM is a very limited resource, and it would be desirable to design tags with the goal of decreasing the number of entries required for membership testing. 
%
%To combat these issues, we present a compression scheme which allows
%the encoding of sequences over a large number of elements, such as service chains or lists of BGP next-hops, into a format easily queried by commodity switches. We show how, with an additional algorithm, this scheme can be used to compress both ordered and unordered sets. Finally, we evaluate our algorithms across both synthetic and real datasets, and show not only does the number of bits needed by our compression scheme compete with the number of bits needed by tags, but that that each core switch need only a constant number of entries per membership test, versus a linear number of entries for the case of flat tagging.


